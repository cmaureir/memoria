\section{Mercado Financiero}
%Reseña a lo que es un mercado financiero.
El mercado financiero es un espacio con marco institucional que permite poner en contacto a oferentes y demandantes para que efectúen 
transacciones financieras. La idea de mercado como foro organizado a la que acuden agentes económicos para efectuar transacciones
queda reducida en el mundo financiero como las bolsas de valores.

El concepto de mercado financiero se utiliza en general para referirse a cualquier mercado organizado en el que se negocien instrumentos financieros
de todo tipo, como por ejemplo, acciones. Además el espacio para generar estas interacciones no necesariamente debe ser físico. Por otro lado el negociar
instrumentos financieros implica a grandes rasgos: intercambiar instrumentos financieros, y definir su precio. Por ende, estos mercados están basados en 
las fuerzas de oferta y demanda, ubicando a todos los oferentes en el mismo lugar para facilitarle la búsqueda a los demandantes.

Una de las razones que hace importante este tipo de mercado, es su funcionalidad, ya que permiten:
\begin{itemize}
	\item Aumentar el capital, siendo esto uno de los casos favorables, ya que también hay probabilidades considerables de disminuir el capital.
	\item Comercio internacional, como en los mercados de divisas, por ejemplo Forex.
	\item Reunir a quienes necesitan recursos financieros, con los que tienen recursos financieros.
\end{itemize}

Se puede encontrar una amplia variedad de tipos de mercados financieros, clasificados por: activos transmitidos, en función a su estructura, según la fase de 
negociación. Para efectos de este trabajo, este estudio se centrará en los mercados de tipo activos transmitidos, más en específico mercados bursátiles.

\subsection{Mercados bursátiles}
Los mercados bursátiles están clasificados en los mercados de capitales, en donde se negocian activos financieros. Este tipo de mercado provee financiamiento
por medio de la emisión de acciones y permiten luego el intercambio de estas. La aplicación más directa de este tipo de mercados, son las bolsas de valores, cuyo
origen se remonta a finales del siglo XV en las ferias medievales de Europa.

Las bolsas de valores se pueden definir como mercados organizados y especializados, en los que se pueden realizar transacciones de títulos de valores por
medio de intermediarios autorizados. Estas bolsas ofrecen al público y a sus miembros facilidades, mecanismos e instrumentos técnicos que facilitan la negociación
de títulos de valores susceptibles de ofertas públicas, a precios determinados mediante subasta.

La principal función de las Bolsas de Valores implican proporcionar a los participantes información verar, objetiva, completa y permanente de los valores
y las empresas inscritas en la Bolsa, sus emisiones y las operaciones que en ella se realicen, supervisión de actividades.

Las componentes de este sistema son los activos, instituciones financieras cuya misión es contactar demandantes y oferentes en los mercados donde se negocian
los diferentes instrumentos o activos financieros.

Dentro de los estudios de la economía, se habla de que este tipo de mercado es de competencia perfecta, es decir posee características 
como: \cite{mankiw2011principles}:
\begin{itemize}
        \item Existe un elevado número de compradores y vendedores. La decisión individual de cada uno de ellos ejercerá escasa influencia sobre el mercado global.
        \item Homogeneidad de los productos, es decir, no existen diferencias entre productos que venden los oferentes.
        \item Transparencia del mercado. TOdos los participantes tienen pleno conocimiento de las condiciones generales en que opera el mercado.
        \item Libertad de entrada y salida de empresas. Todos los participantes, cuando lo deseen, podrán entrar o salir del mercado a costos nulos o casi nulos.
\end{itemize}.

\subsection{Electronic trading}
%Descripción a grandes rasgos de un Electronic market maker.
Al observar el ferviente crecimiento en las tecnologías de información y acceso a ello, se impulsó la creación de los mercados de trading electrónico. Los primeros
mercados de este tipo vieron la luz entre los años 1989 y 1992, pero el principal salto fue que en el año 1997 se dispusieron API's para los usuarios. Si se observa
como aumenta la tecnología y la accesibilidad a ella, era de imaginar los sucesos importantes que seguían en un futuro. Es así como el 2006 nacen las principales
Electronic Communications Networks (ECN): New York Mercantile Exchange (NYMEX), New York Stock Exchange (NYSE), Bolsa de Valores de Santiago, etc. Un ECN es el 
término que se utiliza en los círculos  financieros para un tipo de sistema informático que facilita el comercio de productos financieros fuera de las bolsas de 
valores. Ya en el 2007 con la web 2.0 y los smartphones del mercado, se crean sistemas de trading vía celulares y  se perfilan como una de las mejores alternativas 
para los inversores individuales o minoristas.

Si bien es cierto, cuando se le habla a cualquier persona respecto a las Bolsas de Valores, lo primero que se le viene a la cabeza son una gran cantidad de
personas comprando y vendiendo acciones, y llamando por teléfono de forma desenfrenada. Y en realidad los sistemas en la actualidad han ido dejando de funcionar así
por algunas razones como: Falta de escalabilidad, dificultad para ampliar la base de clientes, insatisfacción del cliente, costos operativos, complejidad de
integración, costo de desarrollar tecnología internamente, etc. Existen bastantes razones por las cuales este tipo de sistemas se vio destinado a desaparecer en el
tiempo, dejando la puerta abierta a los sistemas electrónicos.

Las características de la digitalicación de los mercados mundiales fue impulsada por los siguientes factores:
%Por otro lado la digitalización de los mercados mundiales es impulsada por varios fatores:
\begin{itemize}
        \item La globalización, expansión de redes de telecomunicaciones y con acceso fácil a la información.
        \item Nuevas tecnologías disponibles, más confiables, más seguras y más fáciles de usar.
        \item Los mercados y las empresas de corretaje necesitan crecer y captar nuevos clientes para no perder mercado frente a la competencia.
        \item Los clientes demandan mejor calidad de los servicios a costos más bajos.
        \item Hay una necesidad de contar con mercados más transparentes, más confiables, escalables, sin cuellos de botella y que permitan ser auditados fácilmente.
\end{itemize}

\subsection{Definición}
%Definición de un Electronic market maker
En este documento se abordará el concepto de Electronic Market Maker. Para entender el concepto es necesario definir algunos términos: \cite{glosten1985bid}
\begin{itemize}
        \item Dealers: compran y venden ``activos'' o ``valores''. Lo que buscan es comprar bajo y vender alto.
        \item Se fijan dos precios: \textbf{Bid} (dealer compra, cliente vende). \textbf{Ask} (dealer vende, cliente compra). Lo esperado es que:
                \textbf{ask} $>$ \textbf{bid}. El término \textbf{spread} corresponde a \textbf{ask} - \textbf{bid}, lo que corresponde a la diferencia entre lo que
                el dealer vende y compra (concepto de ganancia). Esto se entiende como la fuente de ingresos de los dealers. \cite{ho1981optimal}
        \item Market order: es una instrucción del cliente al dealer (comprar o vender al mejor precio posible). Asegura la realización de la transacción,
                pero no el precio.
        \item Limit order: es una instrucción del cliente al dealer de pedir una transacción a un precio específico (o más ventajoso).
                No garantiza la realización de la transacción pero si se conoce una cota del precio.
        \item Existe un order book, el cua es un registro de dos colas: una de venta y una de compra. Si una nueva orden entra y no es satisfecha por el limit order,
                esta ingresa al order book (cola de venta o compra).
\end{itemize}

La definición de estos conceptos son recurrentes en la literatura, y son la base con la que los investigadores se expresan.

\subsection{Áreas de investigación}
%Escribir acerca de las áreas de investigación
El Electronic Market Maker se descompone en dos áreas grandes de investigación:
\begin{itemize}
        \item Etablecer el bid-ask spread.
        \item Actualizar el bid-ask spread, de manera predictiva o no predictiva.
\end{itemize}

La primera decisión para el market maker es donde establecer el spread inicial. Es decir encontrar la valoración del activo al ser negociado:
\begin{itemize}
        \item Para un stock de cierto activo, determinar su valor para la compañía. Si el activo corresponde a un bono, buscar el valor presente del pago prometido.
        \item Si no hay mercado establecido, o el mercado tiene poca liquidez, entonces la valorización es la única aproximación. \cite{seppi1997liquidity}
\end{itemize}

La segunda, actualizar el bid-ask, es considerado como el corazón del market maker. Existen formas predictivas al corto plazo teniendo en cuenta el comportamiento
de la curva (bid-ask), esto es posible cuando se tienen valores históricos, y en base a ellos se pueden calcular aproximaciones a los nuevos valores posibles.
El segundo caso posible, es que no se posea información histórica, y que se tenga la información actual del mercado. Estas últimas son inherentemente más simples.

En el estudio de la actualización del bid-ask mediante formas predictivas es donde se utilizan la mayor parte de procedimientos matemáticos. Además en el presente
se trabaja con datos de alta frecuencia, es decir, data proveniente de los servidores con diferencia temporal muy pequeña, lo que deja abierta la posibilidad
de realizar investigación tanto matemática como computacional en la optimización de estos cálculos.
